%designed for PDFLaTeX
\documentclass[]{beamer}
\usetheme{Antibes}%{Frankfurt}%{Berlin}%{Boadilla}

\usepackage[utf8]{inputenc}
\usepackage[english]{babel}
\usepackage{graphicx}
\usepackage{upquote}
\usepackage{listings}
\usepackage{verbatim}
\usepackage{multicol}

\setlength{\parskip}{\medskipamount}
\setlength{\parindent}{0pt}

\title{Comparative Study of Intrinsic Metrics on 3D-Shapes}
\author{Frank Schmidt}
\institute{Computer Vision Group - Technische Universit\"{a}t M\"{u}nchen}
\date{31.10.2014}

%\setbeamertemplate{note page}[plain] %need less gray ink for printing
\setbeameroption{show notes} %comment this for a version without notes

\begin{document}

	\begin{frame}
		\titlepage
	\end{frame}

	\begin{frame}
		\tableofcontents
	\end{frame}

	\begin{frame}
		some stuff about why this is important\\
		picture of 3d-model with different paths as minimal?
	\end{frame}

\section{Background Theory}
\subsection{Shapes as Metric Spaces}
	\begin{frame}
		Metrics (3 equations)\\
		Gromov-Hausdorff distance(Hausdorff in another metric space, perhaps
		with correspondance form, FPS?)
		\note{just trying this out}
	\end{frame}

	\begin{frame}
		\includegraphics[width=\textwidth]{t_pictures/hausdorff} \\
		Hausdorff Distance\\

		some equation for the GH-distance (min H for different embeddings)
	\end{frame}

\subsection{Differential Geometry}
	\begin{frame}
		Regular Surfaces/ First Fundamental form\\
		Multiplication of Jacobian of the maps
		\includegraphics[width=0.8\textwidth]{t_pictures/regular_surface}\\
		Regular surface construction
	\end{frame}

	\begin{frame}
		Laplace-Beltrami operator\\
		why there are eigenfunctions/ how to use them to define other functions
	\end{frame}
% Regular Surfaces, Laplace-Beltrami operator

\section{Intrinsic Metrics}
	\begin{frame}
		euclidean distance? Show example (not intrinsic)
	\end{frame}

\subsection{Geodesic Distance}
	\begin{frame}
		geodesic stuff, telling you that the geodesic is only dependent on the
		first fundamental form; shortest straight distance on the surface\\
		Perhaps another slide to show stuff about the implementation?
	\end{frame}

	\begin{frame}
		\includegraphics[width=\textwidth]{t_pictures/geodesics_new_windows} \\
		How geodesic distance is implemented: using windows and propagation
	\end{frame}

\subsection{Diffusion Distance}
	\begin{frame}
		Diffusion kernel/ Heat equation; \\
		depends on timesteps (2 pics) for local/global preference
	\end{frame}

\subsection{Commute-Time Distance}
	\begin{frame}
		Diffusion distance integrated over all timesteps; tries to combine local
		and global properties
	\end{frame}

\subsection{Biharmonic Distance}
	\begin{frame}
		(Almost) latest invention, tries to combine local and global properties;\\
		Also a pic would be nice probably
	\end{frame}

\section{Testing and Implementation}
\subsection*{}
	\begin{frame}
		3 tests: speed, visually compare plots under different changes of the mesh
		and an error analysis using one to one correspondances\\
		oh, and Farthest point sampling
	\end{frame}

	\begin{frame}
		some implementation stuff i find, otherwise this slide will be not existant
	\end{frame}

\section{Results}
\subsection*{}
	\begin{frame}
		a bunch of pictures of the different metrics, tables and stuff\\
		overall the biharmonic distance is the best
	\end{frame}
%each for itself/together; watch, if my graphics are applicable or need to be made smaller

%conclusion: it was great i did that, because ...

	\begin{frame}
		\centering \large
		Thank you for your attention.\\
		Any Questions?
	\end{frame}

\section*{} %%appendix
\appendix
	\begin{frame}{Sources}
		stuff
	\end{frame}

\end{document}
