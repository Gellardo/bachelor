\chapter{Shapes as Metric Spaces}
\label{chapter:shapeSpaces}
%some stuff about normal distances.
The main question this chapter will be answering is, whether or not there exists such a thing as a ``space of shapes'' and if so, how to discern shapes from each other.
To achieve this, we need the notion of a distance on a shape, in other words, we have to define a metric.

\section{Metrics}
%%%%%%%%%%%%%%%%%%%%%%%%%%%%%%%%%%%%%%%
%%
%%      Metrics
%%
%%%%%%%%%%%%%%%%%%%%%%%%%%%%%%%%%%%%%%%
We start by defining the most important point of this section, the metric space.
\begin{mydef}[metric space]
	A set $M$ is called a metric space, if for each pair of points $x,y \in M$ there is a distance/metric function $d_M: M \times M \rightarrow \real \cup \{0,\infty\}$ such that:
	\begin{itemize}
		\item $d_M(x,y) = 0 \Leftrightarrow x = y$ (identity of indiscernibles)
		\item $d_M(x,y) = d_M(y,x)$ (symmetry)
		\item $d_M(x,y) \le d_M(x,z) + d_M(z,y)\,\forall x,y,z \in M$ (triangle inequality)
	\end{itemize}
\end{mydef}
In other words, if we can find a distance function which fulfills the given properties, every set can be a metric space.
The ones most important to this thesis are the metric spaces defined on all three dimensional shapes.
But what are the interesting properties of such a metric space?
For one, there then is a way to measure how ``close'' one shape is to another by watching its individual points and comparing the behavior of the metric function.
One interesting term in this context is the isometry:
\begin{mydef}
	A surjective, distance-preserving map $f$ is called an isometry, where $f$ being distance preserving means that for $f :X \rightarrow Y$ the equation $d_X(x,y) = d_Y(f(x),f(y)) \,\forall x,y \in X$ holds.
\end{mydef}
Thus, if a shape undergoes a isometric transformation, the distance retained.
The more intuitive way of describing isometries is, that there are two instances of the same shape and one is changed a little, like for example it is rotated or scaled.
There is also the notion of near-isometries for almost isometric shapes, like a human in two different poses (which are not isometric, since there is some bending and stretching of the surface).

% ?:diameter of S, distance of point and set
% details about metric spaces by restriction, ambient spacees with restriction worse than its own intrinsic metric
So if we can define some sort of distance function between shapes, we could distinguish between shapes with small differences (e.g. two humans in different positions) and shapes with big differences like a human and a car.

\section{Gromov-Hausdorff-Distance}
%%%%%%%%%%%%%%%%%%%%%%%%%%%%%%%%%%%%%%%
%%
%%      Gromov-Hausdorff
%%
%%%%%%%%%%%%%%%%%%%%%%%%%%%%%%%%%%%%%%%

