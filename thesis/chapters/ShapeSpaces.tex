\chapter{Shapes as Metric Spaces}
\label{chapter:shapeSpaces}
%some stuff about normal distances.
The main question this chapter will be answering is, whether or not there exists such a thing as a ``space of shapes'' and if so, how to discern shapes from each other.
To achieve this, we need the notion of a distance on a shape, in other words, we have to define a metric.

\section{Metrics}
%%%%%%%%%%%%%%%%%%%%%%%%%%%%%%%%%%%%%%%
%%
%%      Metrics
%%
%%%%%%%%%%%%%%%%%%%%%%%%%%%%%%%%%%%%%%%
\begin{mydef}[metric space]
	A set $M$ is called a metric space, if for each pair of points $x,y \in M$ there is a distance/metric function $d_M: M \times M \rightarrow \real \cup \{0,\infty\}$ such that:
	\begin{itemize}
		\item $d_M(x,y) = 0 \Leftrightarrow x = y$ (identity of indiscernibles)
		\item $d_M(x,y) = d_M(y,x)$ (symmetry)
		\item $d_M(x,y) \le d_M(x,z) + d_M(z,y)\ \forall x,y,z \in M$ (triangle inequality)
	\end{itemize}
\end{mydef}


\section{Gromov-Hausdorff-Distance}
%%%%%%%%%%%%%%%%%%%%%%%%%%%%%%%%%%%%%%%
%%
%%      Gromov-Hausdorff
%%
%%%%%%%%%%%%%%%%%%%%%%%%%%%%%%%%%%%%%%%

