\chapter{Shapes as Metric Spaces}
\label{chapter:shapeSpaces}
%some stuff about normal distances.
The main question this chapter will be answering is, whether or not there exists such a thing as a ``space of shapes'' and if so, how to discern shapes from each other.
To achieve this, we need the notion of a distance on a shape, in other words, we have to define a metric.

\section{Metrics}
%%%%%%%%%%%%%%%%%%%%%%%%%%%%%%%%%%%%%%%
%%
%%      Metrics
%%
%%%%%%%%%%%%%%%%%%%%%%%%%%%%%%%%%%%%%%%
We start by defining the most important point of this section, the metric space.
\begin{mydef}[metric space]
	A set $M$ is called a metric space, if for each pair of points $x,y \in M$ there is a distance/metric function $d_M: M \times M \rightarrow \real_+ \cup \{0,\infty\}$ such that:
	\begin{itemize}
		\item $d_M(x,y) = 0 \Leftrightarrow x = y$ (identity of indiscernibles)
		\item $d_M(x,y) = d_M(y,x)$ (symmetry)
		\item $d_M(x,y) \le d_M(x,z) + d_M(z,y)\,\forall x,y,z \in M$ (triangle inequality)
	\end{itemize}
\end{mydef}
In other words, if we can find a distance function which fulfills the given properties, every set can be a metric space.
The ones most important to this thesis are the metric spaces defined on all three dimensional shapes.
But what are the interesting properties of such a metric space?
For one, there then is a way to measure how ``close'' one shape is to another by watching its individual points and comparing the behavior of the metric function over the shapes.
One interesting term in this context is the isometry:
\begin{mydef}
	A surjective, distance-preserving map $f$ is called an isometry, where $f$ being distance preserving means that for $f :X \rightarrow Y$ the equation $d_X(x,y) = d_Y(f(x),f(y)) \,\forall x,y \in X$ holds.
	Two metric spaces $(X,d_X)$ and $(Y,d_Y)$ are isometric, if there exists such an isometry $f :X \rightarrow Y$.
\end{mydef}
Thus, if a shape undergoes a isometric transformation, the distance retained.
The more intuitive way of describing isometries is, that there are two instances of the same shape and one is changed a little, like for example it is rotated or scaled.
There is also the notion of near-isometries for almost isometric shapes, like a human in two different poses (which are not isometric, since there is some bending and stretching of the surface).

Now there are a few interesting properties of metric spaces.
If $(X,d_X)$ is a metric space and $Y \subset X$, then a metric function for $Y$ can be obtained by the restriction of $d_X$ to the subset $d_Y = \left.d_X\right|_Y$.
This is a useful property because, for example if we wish to compare a part of a surface to the whole, the metric is unchanged on the partial surface.
Moreover, $X$ is called ambient space for $Y$ and even though the restriction of $d_X$ is the simplest, it is not the only way to define a metric on $Y$.
In many cases, there is a more natural intrinsic metric, which means that it is not dependent on the ambient space.
Now, to define the distance of a point $x \in X$ to a subset $Y$, the smallest distance between $x$ and $Y$ is used: $d_X(x,Y) = \inf_{y \in Y} d_X(x,y)$.

\begin{example}
Let us consider the metric space $(\real^2, \|\cdot\|)$ with the euclidean metric and its subset $S \subset \real^2$ containing the points of the unit circle.
The restriction of $\|\cdot\|$ to $S$ would be a possible metric function, but a more intuitive, intrinsic metric is the minimal arc length between two points.
It is easily to be seen, that the minimal arc length is in fact a metric, since it is symmetric, equal to zero if the points are equal and there is no shorter way over a third point and so the triangle inequality is also fulfilled.
Additionally the arc length does not depend on the $\real^2$ coordinates of the points but on their position on the unit circle, what makes the minimal arc length an intrinsic metric.
The two proposed metrics are also not isometric, as the distance between two opposing points $x,y\in S$ is different: $\|x,y\| = 1$ while $min_arclength(x,y) = 2\pi$.
\end{example}
%compactness -> diam < inf
% ?:diameter of S, distance of point and set

So if we can define some sort of distance function between shapes, we could distinguish between shapes with small differences (e.g. two humans in different positions) and shapes with big differences like a human and a car.

\section{Gromov-Hausdorff distance}
%%%%%%%%%%%%%%%%%%%%%%%%%%%%%%%%%%%%%%%
%%
%%      Gromov-Hausdorff
%%
%%%%%%%%%%%%%%%%%%%%%%%%%%%%%%%%%%%%%%%
The most commonly used metric to differentiate three dimensional shapes is the Gromov-Hausdorff distance.
But before we begin working on the Gromov-Huasdorff distance, we need to find a intuition of what it means for surfaces to be ``close'' to each other.
Smooth surfaces in $\real^3$ can be (at least locally) parametrized by a domain $U \subset \real^2$ for example by an embedding $f: U\rightarrow \real^3$.
%TODO picture? surface <- f1 - U - f2-> surface2
Two parametriations then specify a homeomorphism from one surface to the other.
Those two surfaces are said to be ``close'' if the homeomorphism only slightly changes some properties, like distances or derivatives, of the surfaces.
%hausdorff distance -> g-h distance
