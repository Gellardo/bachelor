\clearemptydoublepage

\phantomsection
\addcontentsline{toc}{chapter}{Outline of the Thesis}

\begin{center}
	\huge{Outline of the Thesis}
\end{center}




%--------------------------------------------------------------------
\section*{Part I: Introduction and Background Theory}

\noindent {\scshape Chapter 1: Introduction}  \vspace{1mm}

\noindent  This chapter presents an overview of the thesis and its purpose. Furthermore, it states the motivation of this thesis.  \\

\noindent {\scshape Chapter 2: Shapes as Metric Spaces}  \vspace{1mm}

\noindent  The theoretic foundations of metric spaces are explained. After that, the Gromov-Hausdorff distance as a metric space on three-dimensional surfaces is introduced.   \\

\noindent {\scshape Chapter 3: Differential Geometry}  \vspace{1mm}

\noindent  The basic concepts of the differential geometry of three-dimensional surfaces are introduced. Then the notion of geodesic curves and the resulting geodesic metric are presented.
This chapter finishes after introducing the Laplace operator on regular surfaces and the intrinsic metrics based on the Laplacians eigenfunctions and eigenvalues.\\

%\noindent {\scshape Chapter 4: Applications}  \vspace{1mm}

%\noindent  No thesis without theory.   \\

%--------------------------------------------------------------------
\section*{Part II: Implementation and Experiments}

\noindent {\scshape Chapter 4: Testing Purposes and Pipeline}  \vspace{1mm}

\noindent  The experiments and ideas behind the tests of this paper are presented.
Starting with the timing of the computation of the metrics, we present further experiments to measure their qualitative properties on triangulated meshes.\\

\noindent {\scshape Chapter 5: Implementation Details}  \vspace{1mm}

\noindent  This chapter gives an overview over the decisions and problems that occurred during the course of the implementation. \\

%--------------------------------------------------------------------
\section*{Part III: Results and Conclusion}

\noindent {\scshape Chapter 6: Testing Results}  \vspace{1mm}

\noindent  In this chapter, we describe the results of the experiments and state possible reasons for the performance of the different metrics. \\

\noindent {\scshape Chapter 7: Conclusion and Future Work}  \vspace{1mm}

\noindent This chapter finalizes the thesis by summarizing the results of this thesis and stating possible future research topics.
