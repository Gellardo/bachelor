% Abstract for the TUM report document
% Included by MAIN.TEX


\clearemptydoublepage
\phantomsection
\addcontentsline{toc}{chapter}{Abstract}





\vspace*{2cm}
\begin{center}
{\Large \bfseries Abstract}
\end{center}
\vspace{1cm}

Measuring the distances between pairs of points on a three-dimensional surface is a classical problem in shape analysis.
In this thesis, we compare the most common intrinsic metrics, namely the geodesic, the diffusion, the commute-time and the biharmonic distance.
In order to do so, the concept of metric spaces is introduced and applyed to the space of three-dimensional surfaces.
After presenting the basic concepts of differential geometry on regular surfaces, we define the different metrics first in the continuous setting and then show ways to discretize them to triangular meshes.
We further introduce our experiments to measure the qualitative performance of the intrinsic metrics on different meshes and give some insight into their implementation.
Finally we find, that the biharmonic distance satisfies most of the preferable properties of an intrinsic metric, while the other distances have more specialized strenghts.

\selectlanguage{german}
\vspace*{2cm}
\begin{center}
{\Large \bfseries Zusammenfassung}
\end{center}
\vspace{1cm}

Das Messen von Distanzen zwischen Paaren von Punkten auf dreidimensionalen Oberfl\"achen ist ein klassisches Problem der Shape Analysis.
In dieser Arbeit vergleichen wir die bekanntesten intrinsischen Metriken: die geod\"atische, die Diffusions-, die commute-time and die biharmonische Distanz.
Zu Beginn wird das Konzept von Metrischen R\"aumen eingef\"uhrt und auf den Raum der dreidimensionalen Oberfl\"achen angewendet.
Nachdem die grundlegenden Konzepte der differentiellen Geometrie auf regul\"aren Oberfl\"achen pr\"asentiert wurde, definieren wir die verschiedenen Metriken im kontinuierlichen Raum und diskretisieren sie dann auf Dreiecksnetze.
Im Folgenden werden unsere Experimente zur Messung der qualitativen Performance der Metriken auf verschiedenen Gittern vorgestellt und ein Einblick in ihre Implementierung gew\"ahrt.
Abschlie{\ss}end stellen wir fest, das die biharmonische Distanz die meisten der erw\"unschten Eigenschaften von intrinsischen Metriken erf\"ullt, w\"ahrend die anderen Distanzen spezialisiertere St\"arken haben.
\selectlanguage{english}
